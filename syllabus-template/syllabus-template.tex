\documentclass[12pt, a4paper]{article}
%\usepackage{geometry}
\usepackage[inner=2.0cm,outer=2.0cm,top=2.5cm,bottom=2.5cm]{geometry}
\pagestyle{empty}
\usepackage{graphicx}
\usepackage{fancyhdr, lastpage, bbding, pmboxdraw}
\usepackage[usenames,dvipsnames]{color}
\definecolor{darkblue}{rgb}{0,0,.6}
\definecolor{darkred}{rgb}{.7,0,0}
\definecolor{darkgreen}{rgb}{0,.6,0}
\definecolor{red}{rgb}{.98,0,0}
\usepackage[colorlinks,pagebackref,pdfusetitle,urlcolor=darkblue,citecolor=darkblue,linkcolor=darkred,bookmarksnumbered,plainpages=false]{hyperref}
\renewcommand{\thefootnote}{\fnsymbol{footnote}}
\newcommand{\RNum}[1]{\uppercase\expandafter{\romannumeral #1\relax}}
\pagestyle{fancyplain}
\fancyhf{}
\lhead{ \fancyplain{}{\textsc{SCM 502: Operations \& Supply Chain Management} }}
\chead{ \fancyplain{}{} }
\rhead{ \fancyplain{}{\textsc{Harish Guda} }}
%\rfoot{\fancyplain{}{page \thepage\ of \pageref{LastPage}}}
\fancyfoot[RO, LE] {page \thepage\ of \pageref{LastPage} }
\thispagestyle{plain}

%%%%%%%%%%%% LISTING %%%
\usepackage{listings}
\usepackage{caption}
\DeclareCaptionFont{white}{\color{white}}
\DeclareCaptionFormat{listing}{\colorbox{gray}{\parbox{\textwidth}{#1#2#3}}}
\captionsetup[lstlisting]{format=listing,labelfont=white,textfont=white}
\usepackage{verbatim} % used to display code
\usepackage{fancyvrb}
\usepackage{acronym}
\usepackage{amsthm}
\VerbatimFootnotes % Required, otherwise verbatim does not work in footnotes!
\definecolor{OliveGreen}{cmyk}{0.64,0,0.95,0.40}
\definecolor{CadetBlue}{cmyk}{0.62,0.57,0.23,0}
\definecolor{lightlightgray}{gray}{0.93}
\lstset{
%language=bash,                          % Code langugage
basicstyle=\ttfamily,                   % Code font, Examples: \footnotesize, \ttfamily
keywordstyle=\color{OliveGreen},        % Keywords font ('*' = uppercase)
commentstyle=\color{gray},              % Comments font
numbers=left,                           % Line nums position
numberstyle=\tiny,                      % Line-numbers fonts
stepnumber=1,                           % Step between two line-numbers
numbersep=5pt,                          % How far are line-numbers from code
backgroundcolor=\color{lightlightgray}, % Choose background color
frame=none,                             % A frame around the code
tabsize=2,                              % Default tab size
captionpos=t,                           % Caption-position = bottom
breaklines=true,                        % Automatic line breaking?
breakatwhitespace=false,                % Automatic breaks only at whitespace?
showspaces=false,                       % Dont make spaces visible
showtabs=false,                         % Dont make tabls visible
columns=flexible,                       % Column format
morekeywords={__global__, __device__},  % CUDA specific keywords
}

%%%%%%%%%%%%%%%%%%%%%%%%%%%%%%%%%%%%
\begin{document}
\begin{center}
{\Large \textsc{Course Name}}
\end{center}
\begin{center}
Semester\\
Program \\
School, University\\
\end{center}
%\date{September 26, 2014}

\begin{center}
\rule{\textwidth}{0.4pt}
\begin{minipage}[t]{\textwidth}
\medskip
\begin{tabular}{lll}
\textbf{Instructor} & xxx &  \medskip \\
\textbf{Office} & xxx & \medskip \\
\textbf{Phone} & xxx.xxx.xxxx & \medskip\\
\textbf{Contact} & xxx@xxx & \medskip\\
\textbf{Office Hours} & xxx & \medskip\\
\textbf{Time} & xxx & \medskip \\
\textbf{Venue} &  xxx & \\

\end{tabular}\medskip
\end{minipage}
\rule{\textwidth}{0.4pt}
\end{center}
\vspace{.2cm}
\setlength{\unitlength}{1in}
\renewcommand{\arraystretch}{2}

\noindent\textbf{Course Pages} \begin{enumerate}
\item LMS: All course content such as announcements, slides, homework, grades and other required reading will be made available on LMS. Contact me through messages.
\end{enumerate}

\vskip.15in
\noindent\textbf{Teaching Assistant}
TBD
\begin{enumerate}
  \item Office Hours:
\end{enumerate}


\vskip.15in
\noindent\textbf{Required Material}
You are required to purchase the following material.
\begin{enumerate}
  \item Case Packet
\end{enumerate}

\vskip.15in
\noindent\textbf{Data Requirements} Any supplemental data required to complete homeworks will be provided on Canvas.

\vskip.15in
\noindent\textbf{Supplementary Material (OPTIONAL)} %\footnotemark
This is a restricted list of various interesting and useful resources that will be useful during the course.
\begin{itemize}
\item Free e-book.
\item R and R Studio: Both R and R Studio are available for free online – See \href{https://r4ds.had.co.nz/introduction.html#prerequisites}{here} for details. There are plenty of resources online; please contact me if you have any questions regarding installation and/or use. These software are useful for statistical analysis or simulation.

\end{itemize}

\vskip.15in
\noindent\textbf{Description}  Broadly, there are two modules in this course: \emph{M1} and \emph{M2}.
\begin{itemize}
    \item \textbf{M1} Description.

    \item \textbf{M2} Description.
\end{itemize}

\noindent \textbf{Tentative Course Schedule}
\begin{center}
\begin{minipage}{6.5in}
\begin{flushleft}
The schedule below is tentative. Any unexpected changes to the schedule will be announced in-class (as the course proceeds).

Session 1.1\footnote{Session $w.d$ refers to the session on day $d$ in week $w$, where $d = 1$ (resp., $2$) refers to Monday (resp., Wednesday). For example, Session $1.1$ refers to Monday's session in week 1.} \dotfill Topic \#1  \\
{\color{darkgreen}{\Rectangle}} Idea\#1, Idea\#2, Idea\#3. \smallskip \\	

\end{flushleft}
\end{minipage}
\end{center}

\noindent\textbf{Important Dates}
\begin{center} \begin{minipage}{4in}
\textbf{Homework}
\begin{flushleft}
Session 2.1 \dotfill HW\#1 \medskip \\
Session 3.1 \dotfill HW\#2 \medskip\\
\end{flushleft}
\textbf{Exams}
\begin{flushleft}
Session 4.2 \dotfill Exam \#1 \\
{\color{darkred}{\Rectangle}} An announcement regarding exam policies will be made closer to the exam date. 	\medskip \\
Session 8.1 \dotfill Exam \#2 \\
{\color{darkred}{\Rectangle}} An announcement regarding exam policies will be made closer to the exam date. \medskip \\
\end{flushleft}
\end{minipage}
\end{center}

\vspace*{.15in}
\noindent\textbf{Grading Policy}
Three components determine your grade: Exams, Case Homeworks and Case Presentation. The relative weights are as follows: \medskip\\
\begin{center}
\begin{minipage}{3.5in}
\begin{flushleft}
Exams \dotfill ~ $64$\% \\
{\color{darkred}{\Rectangle}} Two exams, each accounting for $32$\%. 	\medskip \\
Homeworks \dotfill ~ $36$\% \\
{\color{darkred}{\Rectangle}} Six homeworks, each accounting for $6$\%. 	\medskip \\

\end{flushleft}
\end{minipage}
\end{center}
\underline{Grades will be assigned based on a curve fitted to the cumulative final score}. The guidelines of the curve are as follows:
\begin{itemize}
  \item Guideline \#1.
\end{itemize}
\vskip.15in
\noindent\textbf{Learning Objectives}
The key learning objectives of this course are:
\begin{itemize}
\item Objective \#1.
\end{itemize}

\vskip.15in
\noindent\textbf{Other Policies}



%%%%%% THE END
\end{document} 